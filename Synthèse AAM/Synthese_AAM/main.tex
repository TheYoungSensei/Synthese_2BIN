\documentclass{article}[12pt]

%--------------Packages------------------------------
\usepackage[utf8]{inputenc} %Pour encoder du texte en français
\usepackage[francais]{babel} %Pour encoder du texte en français
\usepackage{graphicx} %pour inclure des images
\usepackage{changepage}
\usepackage{version} % permet d'utiliser l'environnement comment
\graphicspath{{./figures/}} %repertoire images
\usepackage{listings} %si on veut afficher du code, le code doit se trouver dans un dossier "codes" 					  %lui même dans le même répertoire que ce fichier tex
\usepackage{color} %nécessaire pour changer les couleurs du highlighting du code
\usepackage{amsmath,amssymb}%pour des maths au cas où
\usepackage{array,multirow,makecell}%Pour manipuler les tableaux
\usepackage{url} %pour utiliser les liens hypertextes
\usepackage{hyperref} %pour utiliser les liens hypertextes
\usepackage{float}
\newlength{\offsetpage}
\setlength{\offsetpage}{2.0cm}
\newenvironment{widepage}{\begin{adjustwidth}{-\offsetpage}{-\offsetpage}%
    \addtolength{\textwidth}{2\offsetpage}}%
{\end{adjustwidth}}

\newcommand{\Java}[2]{
	\begin{itemize}
    	\item[]\lstinputlisting[caption=#2,label=#1]{#1.java}
	\end{itemize}
}
% ---------- Document ------------ %
\begin{document}

\begin{titlepage}

\newcommand{\HRule}{\rule{\linewidth}{0.5mm}} % Defines a new command for the horizontal lines, change thickness here

\center % Center everything on the page
 
%----------------------------------------------------------------------------------------
%	HEADING SECTIONS
%----------------------------------------------------------------------------------------

\textsc{\LARGE Institut Paul Lambin}\\[1.5cm] % Name of your university/college
\textsc{\Large BAC 2 Informatique de gestion}\\[0.5cm] % Major heading such as course name
\textsc{\large Unix}\\[0.5cm] % Minor heading such as course title

%----------------------------------------------------------------------------------------
%	TITLE SECTION
%----------------------------------------------------------------------------------------

\HRule \\[0.4cm]
{ \huge \bfseries Synthèse Unix }\\[0.4cm] % Title of your document
\HRule \\[1.5cm]
 
%----------------------------------------------------------------------------------------
%	AUTHOR SECTION
%----------------------------------------------------------------------------------------

\begin{minipage}{0.4\textwidth}
\begin{flushleft} \large
\emph{Auteurs:}\\
Christopher \textsc{Sacré} \\ % Your name
\end{flushleft}
\end{minipage}
~
\begin{minipage}{0.4\textwidth}
\begin{flushright} \large
\emph{Professeur:} \\
C. \textsc{De Muylder}\\
B. \textsc{Henriet}\\
A. \textsc{Ninane}% Supervisor's Name

\end{flushright}
\end{minipage}\\[4cm]

% If you don't want a supervisor, uncomment the two lines below and remove the section above
%\Large \emph{Author:}\\
%John \textsc{Smith}\\[3cm] % Your name

%----------------------------------------------------------------------------------------
%	DATE SECTION
%----------------------------------------------------------------------------------------

{\large \today}\\[3cm] % Date, change the \today to a set date if you want to be precise

%----------------------------------------------------------------------------------------
%	LOGO SECTION
%----------------------------------------------------------------------------------------

%\includegraphics{Logo}\\[1cm] % Include a department/university logo - this will require the graphicx package
 
%----------------------------------------------------------------------------------------

\vfill % Fill the rest of the page with whitespace

\end{titlepage}

\tableofcontents%table des matières
\newpage
\section{Schémas vus au cours}
\subsection{Semaine x}
\begin{figure}[H]
	\fbox{\includegraphics[scale=0.48]{Schema_X.png}}
    \centering
    \caption{Introduction du cours}
\end{figure}

\section{Informations Supplémentaires}
\subsection{Semaine x}
Si vous souhaitez des informations supllémentaires, plusieurs logiciels ont été mentionné durant ce cours : Hamrmony, Visual Studio Code ainsi que Electron. (Il s'agit d'un cours nous introduisant le concept de notre application).
\end{document}
